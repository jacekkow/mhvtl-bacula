\documentclass[polish]{article}
\usepackage[T1]{fontenc}
\usepackage[utf8]{inputenc}
\usepackage[polish]{babel}
\usepackage{a4wide}
\usepackage{color}
\usepackage{latexsym}
\usepackage[dvips]{graphicx}
\usepackage[dvips]{epsfig}
\usepackage{url}

\definecolor{uxgray}{gray}{0.75}
\newcommand{\uxcmd}[1]{\colorbox{uxgray}{\scalebox{0.6}[0.9]{\texttt{#1}}}}
\newcommand{\ipbox}[0]{\_\_\_.\_\_\_.\_\_\_.\_\_\_}
\newcommand{\tleft}[1]{
\begin{flushleft} 
#1
\end{flushleft}
}

\newenvironment {titemize}{\begin{itemize} \setlength{\itemsep}{-\parsep} } {\end{itemize}} 

\title{\textbf{Instrukcja do ćwiczenia}\\ mhVTL i BareOS}
\author{Piotr Kala \and Jacek Kowalski}
\begin{document}
\maketitle

\begin{tabular}{|l|p{.7\textwidth}|}
\hline
Data wykonania & \\
\hline
Skład grupy & \\
\hline
Ocena & \\
\hline
\end{tabular}

\vspace{0.5cm}

\renewcommand{\labelenumi}{$\Box$\texttt{\theenumi}}
\renewcommand{\labelenumii}{$\Box$\texttt{\theenumii}}

\subsection*{Środowisko}

W skład stanowiska wchodzą dwa komputery, na których uruchomione są
trzy maszyny wirtualne z systemem CentOS w wersji 6 lub 7:

\begin{itemize}
\item Komputer 1 - mhVTL oraz BareOS director
\item Komputer 2 - BareOS client
\end{itemize}

\subsection*{Instalacja mhVTL}

\begin{enumerate}
\item Punkt 1
\item Punkt 2
\end{enumerate}

\subsection*{Instalacja BareOS director}

\begin{enumerate}
\item Punkt 1
\item Punkt 2
\end{enumerate}

\subsection*{Instalacja BareOS client}

\begin{enumerate}
\item Punkt 1
\item Punkt 2
\end{enumerate}

\end{document}
