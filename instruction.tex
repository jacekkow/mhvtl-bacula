\documentclass[polish]{article}
\usepackage[T1]{fontenc}
\usepackage[utf8]{inputenc}
\usepackage[polish]{babel}
\usepackage{a4wide}
\usepackage{color}
\usepackage{latexsym}
\usepackage[dvips]{graphicx}
\usepackage[dvips]{epsfig}
\usepackage{url}

\definecolor{uxgray}{gray}{0.75}
\newcommand{\uxcmd}[1]{\colorbox{uxgray}{\scalebox{0.6}[0.9]{\texttt{#1}}}}
\newcommand{\ipbox}[0]{\_\_\_.\_\_\_.\_\_\_.\_\_\_}
\newcommand{\tleft}[1]{
\begin{flushleft}
#1
\end{flushleft}
}

\newenvironment {titemize}{\begin{itemize} \setlength{\itemsep}{-\parsep} } {\end{itemize}}

\title{\textbf{Instrukcja do ćwiczenia}\\ mhVTL i BareOS}
\author{Piotr Kala \and Jacek Kowalski}
\begin{document}
\maketitle

\begin{tabular}{|l|p{.7\textwidth}|}
\hline
Data wykonania & \\
\hline
Skład grupy & \\
\hline
Ocena & \\
\hline
\end{tabular}

\vspace{0.5cm}

\renewcommand{\labelenumi}{$\Box$\texttt{\theenumi}}
\renewcommand{\labelenumii}{$\Box$\texttt{\theenumii}}


\section{Środowisko}

W skład stanowiska wchodzą dwa komputery, na których uruchomione są
trzy maszyny wirtualne z systemem CentOS w wersji 6 lub 7:

\begin{itemize}
\item Komputer 1, VM 1 - mhVTL oraz BareOS Storage Daemon
\item Komputer 1, VM 2 - BareOS Director
\item Komputer 2 - BareOS File Daemon
\end{itemize}


\section{Przygotowania}

Instrukcja nie obejmuje konfiguracji firewalla.
Poszczególne usługi nasłuchują na następujących portach:

\begin{itemize}
\item mhVTL GUI --- 80,
\item BareOS Director --- 9101,
\item BareOS Storage Daemon --- 9102,
\item BareOS File Daemon --- 9103.
\end{itemize}


\subsection*{Instalacja mhVTL}

\begin{enumerate}
\item Zainstaluj pakiet \texttt{git}:
\begin{verbatim}
yum install git
\end{verbatim}
\item Sklonuj repozytorium ze skryptami do ćwiczenia:
\begin{verbatim}
git clone https://github.com/jacekkow/mhvtl-bacula
\end{verbatim}
\item Uruchom skrypt \texttt{mhvtl.sh}:
\begin{verbatim}
./mhvtl-bacula/mhvtl.sh
\end{verbatim}
\item Po zakończeniu wykonania skryptu uruchom ponownie maszynę wirtualną:
\begin{verbatim}
reboot
\end{verbatim}
\item Uruchom skrypt \texttt{mhvtl\_after\_reboot.sh}:
\begin{verbatim}
./mhvtl-bacula/mhvtl_after_reboot.sh
\end{verbatim}

\end{enumerate}


\subsection*{Instalacja BareOS Storage Daemon}

Instalacja powinna zostać wykonana na tym samym hoście,
na którym zainstalowany został mhVTL.

\begin{enumerate}

\item W przypadku CentOS-a 6 doinstaluj paczkę \texttt{epel-release}:
\begin{verbatim}
yum install epel-release
- LUB -
yum update ca-certificates
yum install https://dl.fedoraproject.org/pub/epel/epel-release-latest-6.noarch.rpm
\end{verbatim}

\item Dodaj repozytorium systemu BareOS - w przypadku CentOS-a 7 popraw wersję w ścieżce:
\begin{verbatim}
curl http://download.bareos.org/bareos/release/latest/CentOS_6/bareos.repo \
 > /etc/yum.repos.d/bareos.repo
\end{verbatim}

\item Zainstaluj paczkę \texttt{bareos-storage-tape}:
\begin{verbatim}
yum install bareos-storage-tape
\end{verbatim}

\end{enumerate}


\subsection*{Instalacja BareOS Director}

\begin{enumerate}

\item W przypadku CentOS-a 6 doinstaluj paczkę \texttt{epel-release}.

\item Dodaj repozytorium systemu BareOS.

\item Zainstaluj paczki \texttt{bareos-director}, \texttt{bareos-bconsole} i \texttt{bareos-database-sqlite3}:
\begin{verbatim}
yum install bareos-director bareos-bconsole bareos-database-sqlite3
\end{verbatim}

\end{enumerate}


\subsection*{Instalacja BareOS File Daemon}

\begin{enumerate}

\item W przypadku CentOS-a 6 doinstaluj paczkę \texttt{epel-release}.

\item Dodaj repozytorium systemu BareOS.

\item Zainstaluj paczkę \texttt{bareos-client}:
\begin{verbatim}
yum install bareos-client
\end{verbatim}

\end{enumerate}


\section{Konfiguracja}


\subsection*{Konfiguracja mhVTL}

\begin{enumerate}

\item W przeglądarce internetowej wejdź pod adres:
\begin{verbatim}
http://adres-VM-z-mhVTL/mhvtl-gui
\end{verbatim}

\item Zaloguj się używając hasła \texttt{mhvtl}

\item W zakładce \texttt{Console} uruchom mhVTL.

\item Upewnij się, że biblioteka taśmowa jest widoczna i dostępna:
\begin{verbatim}
mtx status
\end{verbatim}

\item Wezwij prowadzącego.

\end{enumerate}


\subsection*{Konfiguracja BareOS Director}

\begin{enumerate}

\item Utwórz bazę danych SQLite3 wykonując:
\begin{verbatim}
/usr/lib/bareos/scripts/create_bareos_database
/usr/lib/bareos/scripts/make_bareos_tables
\end{verbatim}

\item Wyedytuj plik \texttt{/etc/bareos/bareos-dir.conf}, konfigurując w nim:
\begin{itemize}
\item adres IP i hasło File Daemona (sekcja \texttt{Client}),
\item job, który będzie tworzył kopie zapasowe wybranego folderu z File Daemona (można wykorzystać \texttt{FileSet SelfTest} jako przykład).
\end{itemize}

\end{enumerate}


\subsection*{Konfiguracja BareOS File Daemon}

\begin{enumerate}

\item Upewnij się, że nazwa i hasło w sekcji \texttt{Director}, a także nazwa w sekcji \texttt{FileDaemon} klienta odpowiadają właściwym wpisom w konfiguracji Directora.

\item Uruchom File Daemona na kliencie:
\begin{verbatim}
service bareos-fd start
\end{verbatim}

\end{enumerate}


\subsection*{Konfiguracja BareOS Storage Daemon}

\begin{enumerate}

\item W pliku konfiguracjyjnym \texttt{/etc/bareos/bareos-sd.conf} dodaj \textbf{jedną} z dostępnych bibliotek taśmowych, ustawiając wszystkie \textbf{powiązane z nią} napędy:
\begin{verbatim}
Autochanger {
  Name = Autochanger
  Device = Drive-0
  ...
  Changer Command = "/usr/lib/bareos/scripts/mtx-changer %c %o %S %a %d"
  Changer Device = /dev/sg11
}

Device {
  Name = Drive-0
  Drive Index = 0
  Media Type = ...
  Archive Device = /dev/nst0
  Autochanger = yes
  LabelMedia = no
  AutomaticMount = yes
  AlwaysOpen = yes
}

...
\end{verbatim}

\item Dodaj analogiczną część (element Storage) w konfiguracji Directora (\texttt{/etc/bareos/bareos-dir.conf}):
\begin{verbatim}
Storage {
  Name = Autochanger
  Address = ...
  Password = ...
  Device = Autochanger
  Media Type = ...
  Autochanger = yes
}
\end{verbatim}

\item Popraw wcześniej utworzony job tak, by kopie zapasowe były zapisywane na taśmach w zmieniaczu (opcja \texttt{Storage =}).

\item Upewnij się, że nazwa i hasło wpisane w sekcji \texttt{Director} w pliku \texttt{bareos-sd.conf} odpowiadają odpowienim wpisom w konfiguracji Directora.

\item Wystartuj usługę \texttt{bareos-sd}:
\begin{verbatim}
service bareos-sd start
\end{verbatim}

\item W razie problemów z uruchomieniem, włącz \texttt{bareos-sd} w trybie debugowania:
\begin{verbatim}
bareos-sd -f
\end{verbatim}
Najczęstszym problemem jest występowanie pliku \texttt{/var/lib/bareos/bareos-sd.9103.pid} po nieoczekiwanym wyłączeniu \texttt{bareos-sd} w związku z błędem składniowym w konfiguracji.

\item Uruchom Directora:
\begin{verbatim}
service bareos-dir start
\end{verbatim}

\item Na serwerze Directora sprawdź poprawność konfiguracji zmieniacza taśm:
\begin{verbatim}
# bconsole
Connecting to Director director:9101
1000 OK: director-dir Version: 15.2.2 (16 November 2015)
Enter a period to cancel a command.
*status slots storage=Autochanger
Automatically selected Catalog: MyCatalog
Using Catalog "MyCatalog"
Connecting to Storage daemon Autochanger at storage:9103 ...
3306 Issuing autochanger "slots" command.
Device "Autochanger" has 105 slots.
Connecting to Storage daemon Autochanger at storage:9103 ...
3306 Issuing autochanger "listall" command.
 Slot |   Volume Name    |   Status  |  Media Type    |         Pool             |
------+------------------+-----------+----------------+--------------------------|
    1*|         M00001L3 |         ? |              ? |                        ? |
...
\end{verbatim}

\item Wezwij prowadzącego.

\end{enumerate}


\subsection*{Kopie zapasowe i przywracanie}

\begin{enumerate}

\item Uruchom wcześniej stworzony job, nazywając odpowiednio taśmę w zmieniaczu:
\begin{verbatim}
*run
A job name must be specified.
The defined Job resources are:
     1: BackupClient
     2: BackupCatalog
     3: RestoreFiles
Select Job resource (1-3): 1
Run Backup job
/ ... /
OK to run? (yes/mod/no): yes
Job queued. JobId=5
You have messages.
*messages
/ ... /
25-May 11:45 localhost-sd JobId 1: Please mount append Volume "Full-0001" or label a new one for:
    Job:          BackupClient.2016-05-25_11.45.47_06
    Storage:      "Drive-0" (/dev/st0)
    Pool:         Full
    Media type:   LTO5
*label
The defined Storage resources are:
     1: File
     2: Autochanger
Select Storage resource (1-2): 2
Select Drive:
     1: Drive 0
     2: ...
1
Enter new Volume name: Full-0001
Enter slot (0 or Enter for none): 15
Defined Pools:
     1: Full
     2: Differential
     3: Incremental
     4: Scratch
Select the Pool (1-4): 1
Connecting to Storage daemon Autochanger at localhost:9103 ...
Sending label command for Volume "Full-0001" Slot 15 ...
3304 Issuing autochanger "load slot 15, drive 0" command.
3305 Autochanger "load slot 15, drive 0", status is OK.
3000 OK label. VolBytes=212 Volume="Full-0001" Device="Drive-0" (/dev/st0)
Requesting to mount Autochanger ...
3001 OK mount requested. Device="Drive-0" (/dev/st0)
*messages
/ ... /
  Termination:            Backup OK
\end{verbatim}

\item Wyładuj taśmę z biblioteki taśmowej i sprawdź czy wróciła do odpowiedniego slotu komendą \texttt{mtx}:

\begin{verbatim}
*unmount
/ .../
Select Drive (1-4): 1
/ ... /
\end{verbatim}

\item Przywróć dane z kopii. Sprawdź, do którego folderu zostaną wgrane i czy Bacula samodzielnie wybierze poprawną taśmę do odzyskiwania.

\begin{verbatim}
*restore
To select the JobIds, you have the following choices:
/ ... /
     3: Enter list of comma separated JobIds to select
/ ... /
Select item:  (1-13): 3
Enter JobId(s), comma separated, to restore: 1
/ ... /
cwd is: /
$ add *
$ done
/ ... /
OK to run? (yes/mod/no): yes
\end{verbatim}

\item Zweryfikuj poprawność przywróconych danych, porównując je z oryginałami.

\item Wezwij prowadzącego.

\end{enumerate}


\section*{Zadania dodatkowe}

\begin{enumerate}

\item Zmień dane w backupowanym folderze, a następnie wykonaj kopię przyrostową i spróbuj przywrócić dane.

\item Przywróć tylko wybrane pliki z kopii.

\end{enumerate}

\end{document}
