\documentclass[polish]{article}
\usepackage[T1]{fontenc}
\usepackage[utf8]{inputenc}
\usepackage[polish]{babel}
\usepackage{a4wide}
\usepackage{color}
\usepackage{latexsym}
\usepackage[dvips]{graphicx}
\usepackage[dvips]{epsfig}
\usepackage{url}

\definecolor{uxgray}{gray}{0.75}
\newcommand{\uxcmd}[1]{\colorbox{uxgray}{\scalebox{0.6}[0.9]{\texttt{#1}}}}
\newcommand{\ipbox}[0]{\_\_\_.\_\_\_.\_\_\_.\_\_\_}
\newcommand{\tleft}[1]{
\begin{flushleft}
#1
\end{flushleft}
}

\newenvironment {titemize}{\begin{itemize} \setlength{\itemsep}{-\parsep} } {\end{itemize}}

\title{\textbf{Instrukcja do ćwiczenia}\\ mhVTL i BareOS}
\author{Piotr Kala \and Jacek Kowalski}
\begin{document}
\maketitle

\begin{tabular}{|l|p{.7\textwidth}|}
\hline
Data wykonania & \\
\hline
Skład grupy & \\
\hline
Ocena & \\
\hline
\end{tabular}

\vspace{0.5cm}

\renewcommand{\labelenumi}{$\Box$\texttt{\theenumi}}
\renewcommand{\labelenumii}{$\Box$\texttt{\theenumii}}


\section{Środowisko}

W skład stanowiska wchodzą dwa komputery, na których uruchomione są
trzy maszyny wirtualne z systemem CentOS w wersji 6 lub 7:

\begin{itemize}
\item Komputer 1 - mhVTL oraz BareOS Director
\item Komputer 2 - BareOS File Daemon
\end{itemize}


\section{Przygotowania}

Instrukcja nie obejmuje konfiguracji firewalla.
Poszczególne usługi nasłuchują na następujących portach:

\begin{itemize}
\item mhVTL GUI --- 80,
\item BareOS Director --- 9101,
\item BareOS Storage Daemon --- 9102,
\item BareOS File Daemon --- 9103.
\end{itemize}


\subsection*{Instalacja mhVTL}

\begin{enumerate}
\item Zainstaluj pakiet \texttt{git}:
\item Sklonuj repozytorium ze skryptami do ćwiczenia:
\begin{verbatim}
git clone https://github.com/jacekkow/mhvtl-bacula
\end{verbatim}
\item Uruchom skrypt \texttt{mhvtl.sh}:
\begin{verbatim}
./mhvtl-bacula/mhvtl.sh
\end{verbatim}
\item Po zakończeniu wykonania skryptu uruchom ponownie maszynę wirtualną:
\item Uruchom skrypt \texttt{mhvtl\_after\_reboot.sh}:
\begin{verbatim}
./mhvtl-bacula/mhvtl_after_reboot.sh
\end{verbatim}

\end{enumerate}

\subsection*{Instalacja BareOS Storage Daemon}

Instalacja powinna zostać wykonana na tym samym hoście,
na którym zainstalowany został mhVTL.

\begin{enumerate}

\item W przypadku CentOS-a 6 doinstaluj paczkę \texttt{epel-release}

\item Dodaj repozytorium systemu BareOS

\item Zainstaluj paczkę \texttt{bareos-storage-tape}

\end{enumerate}

\subsection*{Instalacja BareOS Director}

\begin{enumerate}

\item W przypadku CentOS-a 6 doinstaluj paczkę \texttt{epel-release}

\item Dodaj repozytorium systemu BareOS

\item Zainstaluj paczki \texttt{bareos-director} i \texttt{bareos-database-sqlite3}

\end{enumerate}

\subsection*{Instalacja BareOS File Daemon}

\begin{enumerate}

\item W przypadku CentOS-a 6 doinstaluj paczkę \texttt{epel-release}

\item Dodaj repozytorium systemu BareOS
\begin{verbatim}
curl http://download.bareos.org/bareos/release/latest/CentOS_6/bareos.repo \
 > /etc/yum.repos.d/bareos.repo
\end{verbatim}

\item Zainstaluj paczkę bareos-client

\end{enumerate}


\section{Konfiguracja}


\subsection*{Konfiguracja mhVTL}

\begin{enumerate}

\item W przeglądarce internetowej wejdź pod adres
\begin{verbatim}
http://adres-VM-z-mhVTL/mhvtl-gui
\end{verbatim}

\item Zaloguj się używając hasła \texttt{mhvtl}

\item W GUI wybierz zakładkę \texttt{Setup} i opcję \texttt{Add}

\item Ustaw konfigurację \texttt{Standard} i markę \texttt{IBM}

\item Wybierz model biblioteki \texttt{03584L22}, napędu \texttt{ULT3580-TD3} i rodzaj taśm \texttt{LTO3}

\item Gdy system zapyta, zgódź się na restart usług mhvtl i w razie potrzeby wystartuj je

\item Wezwij prowadzącego

\end{enumerate}


\subsection*{Konfiguracja BareOS Director}

\begin{enumerate}

\item Utwórz bazę danych SQLite3 wykonując

\item Wyedytuj plik \texttt{/etc/bareos/bareos-dir.conf}, konfigurując w nim
\begin{itemize}
\item adres IP i hasło File Daemona (sekcja \texttt{Client})
\item job, który będzie tworzył kopie zapasowe wybranego folderu z File Daemona (można wykorzystać \texttt{FileSet SelfTest} jako przykład)
\end{itemize}

\item Zrestartuj usługę

\end{enumerate}


\subsection*{Konfiguracja BareOS Storage Daemon}

\begin{enumerate}

\item W pliku konfiguracjyjnym \texttt{/etc/bareos/bareos-sd.conf} dodaj ustawioną wcześniej wirtualną bibliotekę taśmową

\item Dodaj analogiczną część (element Storage) w konfiguracji Directora (\texttt{/etc/bareos/bareos-dir.conf})

\item Popraw wcześniej utworzony job tak, by kopie zapasowe były zapisywane na taśmach w zmieniaczu (opcja \texttt{Storage =})

\item Wystartuj usługę \texttt{bareos-sd}

\item Uruchom Directora

\item Na serwerze Directora sprawdź poprawność konfiguracji zmieniacza taśm

\item Wezwij prowadzącego

\end{enumerate}

\section*{Kopie zapasowe i przywracanie}

\begin{enumerate}

\item Wykonaj backup przyrostowy (i opcjonalnie pełny). Wyniki zapisz w tabeli poniżej

\item Uruchom wcześniej stworzony job, nazywając odpowiednio taśmę w zmieniaczu
\begin{verbatim}
*run
A job name must be specified.
The defined Job resources are:
     1: BackupClient
     2: BackupCatalog
     3: RestoreFiles
Select Job resource (1-3): 1
Run Backup job
/ ... /
OK to run? (yes/mod/no): yes
Job queued. JobId=5
You have messages.
*messages
/ ... /
25-May 11:45 localhost-sd JobId 1: Please mount append Volume "Full-0001" or label a new one for:
    Job:          BackupClient.2016-05-25_11.45.47_06
    Storage:      "Drive-0" (/dev/st0)
    Pool:         Full
    Media type:   LTO3
*label
The defined Storage resources are:
     1: File
     2: Autochanger
Select Storage resource (1-2): 2
Select Drive:
     1: Drive 0
     2: ...
1
Enter new Volume name: Full-0001
Select slot:
15
Defined Pools:
     1: Full
     2: Differential
     3: Incremental
     4: Scratch
Select the Pool (1-4): 1
Connecting to Storage daemon Autochanger at localhost:9103 ...
Sending label command for Volume "Full-0001" Slot 15 ...
3304 Issuing autochanger "load slot 15, drive 0" command.
3305 Autochanger "load slot 15, drive 0", status is OK.
3000 OK label. VolBytes=212 Volume="Full-0001" Device="Drive-0" (/dev/st0)
Requesting to mount Autochanger ...
3001 OK mount requested. Device="Drive-0" (/dev/st0)
*messages
/ ... /
  Termination:            Backup OK
\end{verbatim}

\item Usuń foldery, dla których została utworzona kopia zapasowa

\item Przywróć dane z kopii. Sprawdź, do którego folderu zostają wgrane

\item Zweryfikuj poprawność danych przywróconych z taśmy

\item Wezwij prowadzącego

\end{enumerate}

\end{document}
