\documentclass[polish]{article}
\usepackage[T1]{fontenc}
\usepackage[utf8]{inputenc}
\usepackage[polish]{babel}
\usepackage{a4wide}
\usepackage{color}
\usepackage{latexsym}
\usepackage[dvips]{graphicx}
\usepackage[dvips]{epsfig}
\usepackage{url}

\definecolor{uxgray}{gray}{0.75}
\newcommand{\uxcmd}[1]{\colorbox{uxgray}{\scalebox{0.6}[0.9]{\texttt{#1}}}}
\newcommand{\ipbox}[0]{\_\_\_.\_\_\_.\_\_\_.\_\_\_}
\newcommand{\tleft}[1]{
\begin{flushleft}
#1
\end{flushleft}
}

\newenvironment {titemize}{\begin{itemize} \setlength{\itemsep}{-\parsep} } {\end{itemize}}

\title{\textbf{Instrukcja do ćwiczenia}\\ mhVTL i BareOS}
\author{Piotr Kala \and Jacek Kowalski}
\begin{document}
\maketitle

\begin{tabular}{|l|p{.7\textwidth}|}
\hline
Data wykonania & \\
\hline
Skład grupy & \\
\hline
Ocena & \\
\hline
\end{tabular}

\vspace{0.5cm}

\renewcommand{\labelenumi}{$\Box$\texttt{\theenumi}}
\renewcommand{\labelenumii}{$\Box$\texttt{\theenumii}}


\section{Środowisko}

W skład stanowiska wchodzą dwa komputery, na których uruchomione są
trzy maszyny wirtualne z systemem CentOS w wersji 6 lub 7:

\begin{itemize}
\item Komputer 1, VM 1 - mhVTL oraz BareOS Storage Daemon
\item Komputer 1, VM 2 - BareOS Director
\item Komputer 2 - BareOS File Daemon
\end{itemize}


\section{Przygotowania}

Instrukcja nie obejmuje konfiguracji firewalla.
Poszczególne usługi nasłuchują na następujących portach:

\begin{itemize}
\item mhVTL GUI --- 80,
\item BareOS Director --- 9101,
\item BareOS Storage Daemon --- 9102,
\item BareOS File Daemon --- 9103.
\end{itemize}


\subsection*{Instalacja mhVTL}

\begin{enumerate}
\item Zainstaluj pakiet \texttt{git}:

\item Sklonuj repozytorium ze skryptami do ćwiczenia:
\begin{verbatim}
https://github.com/jacekkow/mhvtl-bacula
\end{verbatim}

\item Uruchom skrypt \texttt{mhvtl.sh} z repozytorium.

\item Po zakończeniu wykonania uruchom ponownie maszynę wirtualną.

\item Uruchom skrypt \texttt{mhvtl\_after\_reboot.sh}.

\end{enumerate}

\subsection*{Instalacja BareOS Storage Daemon}

Instalacja powinna zostać wykonana na tym samym hoście,
na którym zainstalowany został mhVTL.

\begin{enumerate}

\item W przypadku CentOS-a 6 doinstaluj repozytorium EPEL.

\item Dodaj repozytorium systemu BareOS - w przypadku CentOS-a 7 popraw wersję w ścieżce:
\begin{verbatim}
curl http://download.bareos.org/bareos/release/latest/CentOS_6/bareos.repo \
 > /etc/yum.repos.d/bareos.repo
\end{verbatim}

\item Zainstaluj paczkę \texttt{bareos-storage-tape}.

\end{enumerate}

\subsection*{Instalacja BareOS Director}

\begin{enumerate}

\item W przypadku CentOS-a 6 doinstaluj repozytorium EPEL.

\item Dodaj repozytorium systemu BareOS.

\item Zainstaluj paczki \texttt{bareos-director}, \texttt{bareos-bconsole} i \texttt{bareos-database-sqlite3}.

\end{enumerate}

\subsection*{Instalacja BareOS File Daemon}

\begin{enumerate}

\item W przypadku CentOS-a 6 doinstaluj repozytorium EPEL.

\item Dodaj repozytorium systemu BareOS.

\item Zainstaluj paczkę \texttt{bareos-client}.

\end{enumerate}


\section{Konfiguracja}


\subsection*{Konfiguracja mhVTL}

\begin{enumerate}

\item W przeglądarce internetowej wejdź pod adres:
\begin{verbatim}
http://adres-VM-z-mhVTL/mhvtl-gui
\end{verbatim}

\item Zaloguj się używając hasła \texttt{mhvtl}

\item W zakładce \texttt{Console} uruchom mhVTL.

\item Upewnij się, że biblioteka taśmowa jest widoczna i dostępna używając polecenia \texttt{mtx}.

\item Wezwij prowadzącego.

\end{enumerate}


\subsection*{Konfiguracja BareOS Director}

\begin{enumerate}

\item Zainicjuj bazę danych SQLite3 dla systemu BareOS.

\item Wyedytuj plik \texttt{/etc/bareos/bareos-dir.conf}, konfigurując w nim:
\begin{itemize}
\item adres IP i hasło File Daemona (sekcja \texttt{Client}),
\item job, który będzie tworzył kopie zapasowe wybranego folderu z File Daemona (można wykorzystać \texttt{FileSet SelfTest} jako przykład).
\end{itemize}

\end{enumerate}


\subsection*{Konfiguracja BareOS File Daemon}

\begin{enumerate}

\item Upewnij się, że nazwa i hasło w sekcji \texttt{Director}, a także nazwa w sekcji \texttt{FileDaemon} klienta odpowiadają właściwym wpisom w konfiguracji Directora.

\item Uruchom File Daemona na kliencie - usługa nazywa się \texttt{bareos-fd}.

\end{enumerate}


\subsection*{Konfiguracja BareOS Storage Daemon}

\begin{enumerate}

\item W pliku konfiguracjyjnym \texttt{/etc/bareos/bareos-sd.conf} dodaj \textbf{jedną} z dostępnych bibliotek taśmowych, ustawiając wszystkie \textbf{powiązane z nią} napędy.

\item Dodaj analogiczną część (element Storage) w konfiguracji Directora (\texttt{/etc/bareos/bareos-dir.conf}).

\item Popraw wcześniej utworzony job tak, by kopie zapasowe były zapisywane na taśmach w zmieniaczu (opcja \texttt{Storage =}).

\item Upewnij się, że nazwa i hasło wpisane w sekcji \texttt{Director} w pliku \texttt{bareos-sd.conf} odpowiadają odpowienim wpisom w konfiguracji Directora.

\item Wystartuj usługę \texttt{bareos-sd}.

\item Uruchom Directora - usługa \texttt{bareos-dir}.

\item Na serwerze Directora sprawdź poprawność konfiguracji zmieniacza taśm używając w konsoli systemu BareOS komendy \texttt{status slots}.

\item Wezwij prowadzącego.

\end{enumerate}


\section{Kopie zapasowe i przywracanie}

\begin{enumerate}

\item Przygotuj pliki, na których będzie wykonywana kopia zapasowa.

\item Wykonaj backup pełny w zmieniaczu taśmowym, nazywając odpowiednio taśmę. Zapisz czas wykonania i rozmiar kopii:

\begin{tabular}{|c|c|}
  \hline
  czas wykonania & rozmiar \\
  \hline
  & \\
  \hline
\end{tabular}

\item Wyładuj taśmę z biblioteki taśmowej używając \texttt{bconsole} i sprawdź czy wróciła do odpowiedniego slotu komendą \texttt{mtx}.

\item Zmodyfikuj pliki w backupowanym katalogu i wykonaj kopię przyrostową. Zapisz czas wykonania i rozmiar kopii:

\begin{tabular}{|c|c|}
  \hline
  czas wykonania & rozmiar \\
  \hline
  & \\
  \hline
\end{tabular}

\item Przywróć dane z kopii. Sprawdź, do którego folderu zostaną wgrane i czy Bacula samodzielnie wybierze poprawną taśmę do odzyskiwania.

\item Zweryfikuj poprawność przywróconych danych.

\item Wezwij prowadzącego.

\end{enumerate}

\end{document}
